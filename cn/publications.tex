\section{已发表论文}
% AGU style: https://publications.agu.org/agu-grammar-and-style-guide/
\newcommand{\Revision}{\emph{正在审稿}}
\newcommand{\CS}{*} % 通讯作者
\newcommand{\CF}{\textsuperscript{\#}} % 共同一作

\CS 通讯作者,\CF 共同一作
\begin{etaremune}
\item \Me, \SWei, \WWang, \& \FWang, (2022).
    Stress drops of intermediate-depth and deep earthquakes in the Tonga slab.
    \emph{Journal of Geophysical Research: Solid Earth},
    \emph{in press}
\item \JYao, \Me, \LSun, \& \LWen\ (2021).
    Comment on ``Origin of temporal changes of inner-core seismic waves'' by Yang and Song (2020).
    \emph{Earth and Planetary Science Letters}, \emph{553}, 116640.
    \DOI{10.1016/j.epsl.2020.116640}
\item \SWei, \PShearer, \CLithgowBertelloni, \LStixrude, \& \Me\ (2020).
    Oceanic plateau of the Hawaiian mantle plume head subducted to the uppermost lower mantle.
    \emph{Science}, \emph{370}, 983--987.
    \DOI{10.1126/science.abd0312}
\item \Me\CS, \MLv, \SWei, \SDorfman, \& \PShearer\ (2020).
    Global variations of Earth's 520- and 560-km discontinuities.
    \emph{Earth and Planetary Science Letters}, \emph{552}, 116600. \\
    \DOI{10.1016/j.epsl.2020.116600}
\item
    \PWessel, \JLuis, \LUieda, \RScharroo, \FWobbe, \WSmith, \& \Me\ (2019).
    The Generic Mapping Tools Version 6.
    \emph{Geochemistry, Geophysics, Geosystems}, \emph{20}(11), 5556--5564.
    \DOI{10.1029/2019GC008515}
\item
    \JYao, \Me, \LSun, \& \LWen\ (2019).
    Temporal change of seismic Earth's inner core phases: inner core differential rotation or temporal change of inner core surface?
    \emph{Journal of Geophysical Research: Solid Earth}, \emph{124}(7), 6720--6736.
    \DOI{10.1029/2019JB017532}
\item
    \WFan, \SWei, \Me, \JMcGurie, \& \DWiens\ (2019).
    Complex and diverse rupture processes of the 2018 Mw 8.2 and Mw 7.9 Tonga-Fiji deep earthquakes.
    \emph{Geophysical Research Letters}, \emph{46}(5), 2434--2448.
    \DOI{10.1029/2018GL080997}
\item
    \JYao, \Me\CF, \ZLu, \LSun, \& \LWen\ (2018).
    Triggered seismicity after North Korea's 3 September 2017 nuclear test.
    \emph{Seismological Research Letters}, \emph{89}(6), 2085--2093.
    \DOI{10.1785/0220180135}
\item
    \JYao, \Me\CF, \LSun, \& \LWen\ (2018).
    Source characteristics of North Korea's 3 September 2017 nuclear test.
    \emph{Seismological Research Letters}, \emph{89}(6), 2078--2084.
    \DOI{10.1785/0220180134}
\item
    \Me\CS, \JYao, \& \LWen\ (2018).
    Collapse and earthquake swarm after North Korea's 3 September 2017 nuclear test.
    \emph{Geophysical Research Letters}, \emph{45}(9), 3976--3983.
    \DOI{10.1029/2018GL077649}
\item
    温联星, \textbf{田冬冬}, 姚家园 (2018).
    地球内核及其边界的结构特征和动力学过程.
    \emph{地球物理学报}, \emph{61}(3), 803--818.
    \DOI{10.6038/cjg2018L0500}
\item
    \Me, \& \LWen\ (2017).
    Seismological evidence for a localized mushy zone at the Earth's inner core boundary.
    \emph{Nature Communications}, 8, 165.
    \DOI{10.1038/s41467-017-00229-9}
\item
    \XChen, \Me, \& \LWen\ (2015).
    Microseismic sources during hurricane sandy.
    \emph{Journal of Geophysical Research: Solid Earth}, \emph{120}(9), 6386--6403.
    \DOI{10.1002/2015JB012282}
\item \MZhang, \Me, \& \LWen\ (2014).
    A new method for earthquake depth determination: stacking multiple-station autocorrelograms.
    \emph{Geophysical Journal International}, \emph{197}(2), 1107--1116.
    \DOI{10.1093/gji/ggu044}
\end{etaremune}


%\subsection*{尚未发表论文(审稿中/修改中)}
%\begin{etaremune}
%\end{etaremune}

