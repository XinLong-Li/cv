\documentclass[11pt, a4paper]{article}
\usepackage[fontset=fandol]{ctex}  % 中文支持
\usepackage[left=1in,right=1in,top=1in,bottom=1in]{geometry}  % 设置页面边距
\usepackage{fontspec}  % 设置字体
\usepackage[default,semibold]{sourcesanspro}
\usepackage{sourcecodepro}
\usepackage{etaremune}  % 对列表各项逆序编号(用于对文章进行编号)

% 个人信息
\newcommand{\Title}{学术简历}
\newcommand{\FirstName}{Dongdong}
\newcommand{\LastName}{Tian}
\newcommand{\Initials}{D}
\newcommand{\MyName}{田冬冬}
\newcommand{\MyRole}{特任教授}
\newcommand{\Tian}{\textbf{\LastName, \Initials.}}  % For citations
\newcommand{\Email}{dtian@cug.edu.cn}
\newcommand{\Website}{me.seisman.info}
\newcommand{\Phone}{}
\newcommand{\Affiliation}{中国地质大学(武汉)\hspace{1ex} 地球物理与空间信息学院}
\newcommand{\Address}{湖北省武汉市洪山区鲁磨路 388 号基委楼 205 室}

% 合作者
\newcommand{\XChen}{Chen, X.}
\newcommand{\SDorfman}{Dorfman, S. M.}
\newcommand{\WFan}{Fan, W.}
\newcommand{\CLithgowBertelloni}{Lithgow-Bertelloni, C.}
\newcommand{\ZLu}{Lu, Z.}
\newcommand{\JLuis}{Luis, J.}
\newcommand{\MLv}{Lv, M.}
\newcommand{\JMcGurie}{McGuire, J. J.}
\newcommand{\RScharroo}{Scharroo, R.}
\newcommand{\PShearer}{Shearer, P. M.}
\newcommand{\WSmith}{Smith, W. H. F.}
\newcommand{\LStixrude}{Stixrude, L.}
\newcommand{\LSun}{Sun, L.}
\newcommand{\LUieda}{Uieda, L.}
\newcommand{\WWang}{Wang, W.}
\newcommand{\SWei}{Wei, S. S.}
\newcommand{\LWen}{Wen, L.}
\newcommand{\PWessel}{Wessel, P.}
\newcommand{\DWiens}{Wiens, D. A.}
\newcommand{\FWobbe}{Wobbe, F.}
\newcommand{\JYao}{Yao, J.}
\newcommand{\MZhang}{Zhang, M.}

% 控制文字字体
\usepackage{anyfontsize}

% 一些命令
\newcommand{\DOI}[1]{doi:\href{https://dx.doi.org/#1}{#1}}

% 以“年/月”格式显示日期
\usepackage{datetime}
\newdateformat{monthyear}{\THEYEAR/\THEMONTH}

% 设置每节标题的前后空白
\usepackage{titlesec}
\titlespacing*{\section}{0pt}{1ex}{1ex}
% 设置 section 不显示编号,且可生成目录
\titleformat{\section}{\normalfont\Large\bfseries}{}{0pt}{}

% 设置行间距
%\renewcommand{\baselinestretch}{1.1}
\renewcommand{\arraystretch}{1.4}
\setlength{\parindent}{0pt} % no indent for paragraph

% 减小列表中各项之间的间距
\usepackage{enumitem}
\setlist{itemsep=0.2em}

% 设置页眉页脚
\usepackage{fancyhdr}
% 页眉
\pagestyle{fancy}
\fancyhf{}
\chead{
    \itshape
    \fontsize{10pt}{12pt}\selectfont
    \MyName
    \hspace{0.2cm} -- \hspace{0.2cm}
    \Title
    \hspace{0.2cm} -- \hspace{0.2cm}
    \monthyear\today
}
\rhead{}
% 页脚
\cfoot{\fontsize{10pt}{0}\selectfont \thepage}
\renewcommand{\headrulewidth}{0pt}

% 使用自定义颜色
\usepackage[usenames,dvipsnames]{xcolor}
\definecolor{MarkerColour}{HTML}{B6073F}
\newcommand{\makefield}[2]{\makebox[1.5em]{\color{MarkerColour!80!black}#1} #2}

% PDF 元信息以及超链接
\usepackage[colorlinks=true]{hyperref}
\hypersetup{ % document metadata
    pdftitle = {\MyName\ - \Title},
    pdfauthor = {\MyName},
    linkcolor=black,
    citecolor=black,
    filecolor=black,
    urlcolor=MidnightBlue,
}

\begin{document}
% 首页不显示页眉页脚
\thispagestyle{empty}

% CV 封面页
\begin{center}
    \kaishu
    {\fontsize{28pt}{0}\selectfont \MyName}
    \\[0.5cm]
    {\fontsize{17pt}{0}\selectfont \MyRole}
    \\[0.3cm]
    {\fontsize{13pt}{0}\selectfont
        \Affiliation
        \\[0.2cm]
        \Address
        \\[0.08cm]
        %\makefield{\faEnvelopeO}{\href{mailto:\Email}{\texttt{\Email}}}
        %\, | \,
        %\makefield{\faGlobe}{\url{\Website}}
        邮箱: \href{mailto:\Email}{\texttt{\Email}}
        \, | \,
        主页: \href{https://\Website}{\Website}
    }
\end{center}

\section{Education}

\begin{EntriesTable}{0.05}{0.02}{0.93}
2018 & Ph.D in Geophysics, University of Science and Technology of China (USTC), Hefei, China \\
2012 & B.S. in Geophysics, University of Science and Technology of China (USTC), Hefei, China \\
\end{EntriesTable}

\section*{工作经历}

\begin{tabular}{p{0.18\textwidth} p{0.82\textwidth}}
2021/11 至今 & 特任教授,中国地质大学(武汉),中国湖北省武汉市 \\
2018/08--2021/09 & 博士后研究助理,密西根州立大学,美国密西根州东兰辛 \\
\end{tabular}

\section*{研究方向及兴趣}

\begin{itemize}
\item 地球深部结构
\item 地震发生机制
\item 地震波传播理论
\end{itemize}

\section{Professional Societies \& Activities}

\begin{itemize}
\item Member of the \href{https://sites.agu.org/}{American Geophysical Union (AGU)} (2012--present)
\item Peer-reviewer of scientific journals:
      \emph{Geophysical Research Letters},
      \emph{Seismological Research Letters},
      \emph{Review of Scientific Instruments},
      \emph{Journal of Open Source Software},
      \emph{Results in Geophysical Sciences}
\item Founder of the \href{https://blog.seisman.info}{SeisMan blog} (since 2013),
      \href{http://gmt-china.org/}{GMT China Community} (since 2016)
      and \href{https://seismo-learn.org/}{seismo-learn} (since 2020)
\item Core developer of the \href{https://github.com/GenericMappingTools/gmt}{Generic Mapping Tools (GMT)} and
	  \href{https://github.com/GenericMappingTools/pygmt}{PyGMT} (2018--present)
\item Research assistant and database manager for \href{http://chinageorefmodel.org/}{China Seismological Reference Model} (2016--2018)
\item Judge for the Outstanding Student Paper Award, AGU Fall Meeting (2018--2020)
\end{itemize}

\section*{荣誉}

\begin{tabular}{p{0.05\textwidth} p{0.95\textwidth}}
2018 & 中国科学院院长奖 \\
2018 & 中国科学技术大学优秀毕业生 \\
2017 & 中国地球科学联合学术年会 优秀学生论文奖 \\
2017 & 博士生国家奖学金 \\
2014 & 光华奖学金 \\
2010 & 光华奖学金 \\
2009 & 中国科学技术大学优秀志愿者 \\
\end{tabular}

\section{Received Funds}

\begin{itemize}
\item Startup, CUG One Hundred Talents Program, \textyen\ 2,000k (2021--2026)
\end{itemize}

% AGU style: https://publications.agu.org/agu-grammar-and-style-guide/
\newcommand{\Revision}{\textit{under revision}}
\newcommand{\CS}{*} % corresponding author
\newcommand{\CF}{\textsuperscript{\#}} % co-first author

\section*{Peer-reviewed Publications}
\CS corresponding author, \CF co-first author.

\begin{etaremune}
\item
    Yao, J., \Tian, Sun, L., \& Wen, L.
    Temporal change of seismic Earth's inner core phases: inner core differential rotation or temporal change of inner core surface?
    \textit{Journal of Geophysical Research: Solid Earth},
    \DOI{10.1029/2019JB017532}.
    \textit{in press}.
\item
    Fan, W., S.S. Wei, \Tian, J.J. McGuire, and D.A. Wiens. (2019).
    Complex and diverse rupture processes of the 2018 Mw 8.2 and Mw 7.9 Tonga-Fiji deep earthquakes.
    \textit{Geophysical Research Letters}, \textit{46}(5), 2434--2448.
    \DOI{10.1029/2018GL080997}
\item
    Yao, J., \Tian\CF, Lu, Z., Sun, L., \& Wen, L. (2018).
    Triggered seismicity after North Korea's 3 September 2017 nuclear test.
    \textit{Seismological Research Letters}, \textit{89}(6), 2085--2093.
    \DOI{10.1785/0220180135}
\item
    Yao, J., \Tian\CF, Sun, L., \& Wen, L. (2018).
	Source characteristics of North Korea's 3 September 2017 nuclear test.
    \textit{Seismological Research Letters}, \textit{89}(6), 2078--2084.
    \DOI{10.1785/0220180134}
\item
    \Tian, Yao, J., \& Wen, L. (2018).
    Collapse and earthquake swarm after North Korea's 3 September 2017 nuclear test.
    \textit{Geophysical Research Letters}, \textit{45}(9), 3976--3983.
    \DOI{10.1029/2018GL077649}
\item
    Wen, L., \Tian, \& Yao, J. (2018).
    Seismic structure and dynamic process of the Earth's inner core and its boundary.
    \textit{Chinese Journal of Geophysics}, \textit{61}(3), 803--818.
    \DOI{10.6038/cjg2018L0500} [in Chinese]
\item
    \Tian, \& Wen, L. (2017).
    Seismological evidence for a localized mushy zone at the Earth's inner core boundary.
    \textit{Nature communications}, 8, 165.
    \DOI{10.1038/s41467-017-00229-9}
\item
    Chen, X., \Tian, \& Wen, L. (2015).
    Microseismic sources during hurricane sandy.
    \textit{Journal of Geophysical Research: Solid Earth}, \textit{120}(9), 6386--6403.
    \DOI{10.1002/2015JB012282}
\item Zhang, M., \Tian, \& Wen, L. (2014).
    A new method for earthquake depth determination: stacking multiple-station autocorrelograms.
    \textit{Geophysical Journal International}, \textit{197}(2), 1107--1116.\\
    \DOI{10.1093/gji/ggu044}
\end{etaremune}

\subsection*{Papers submitted/under revision}
\begin{etaremune}
\item
    Wessel, P., Luis, J., Uieda, L., Scharroo, R., Wobbe, F., Smith, W. H. F., \& \Tian,
    The Generic Mapping Tools, Version 6.
    \textit{submitted to Geochemistry, Geophysics, Geosystems}.
\end{etaremune}

\subsection*{Papers in Preparation}
\begin{etaremune}
\item
    \Tian, \& Wen, L.
    Improved relative moment tensor inversion method and applications to clusters of small earthquakes.
\item
    \Tian, \& Wen, L.
    Three types of Earth's inner core boundary.
\item
    \Tian, \& Wen, L.
    Simulating wave propagation in a faulted medium using a 3D finite difference method.
\end{etaremune}

\section*{Meeting Abstracts}

\begin{etaremune}
\item Wei, S. S., Shearer, P. M., Lithgow-Bertelloni, C., Stixrude, L., \& \Tian\ (2021).
	Oceanic plateau of the Hawaiian mantle plume head subducted to the uppermost lower mantle,
	Abstract EGU21-13874 virtually presented at EGU General Assembly 2021.
\item \Tian, Wang, W., Wang, F., \& Wei, S. S. (2020).
	Source spectra of intermediate-depth and deep earthquakes in the Tonga subduction zone.
	Abstract S054-0012 virtually presented at 2020 AGU Fall Meeting.
\item Wei, S. S., \Tian, Shearer, P. M., Lv, M., Dorfman S. M., Lithgow-Bertelloni, C. R., \& Stixrude, L. P. (2020).
	Compositional heterogeneities in the mid-mantle revealed by seismic discontinuities and reflectors.
	Abstract DI016-0008 virtuablly presented at 2020 AGU Fall Meeting.
\item \Tian, Wang, W. \& Wei, S. S. (2019).
	Source spectra and stress drop of deep earthquakes in the Tonga subduction zone.
	Abstract S13C-0458 presented at 2019 AGU Fall Meeting, San Francisco, CA, USA.
\item
    \Tian, Wei, S. S., \& Shearer, M. P. (2019).
    Global variations of the 520-km discontinuity.
    Presented at Gordon Research Conference: Interior of the Earth, South Hadley, MA, USA.
\item
    \Tian, Wei, S. S., \& Shearer, M. P. (2018).
    Global variations of the 520-km discontinuity.
    Abstract DI31C-0024 presented at 2018 AGU Fall Meeting, Washington, DC, USA.
\item
    \Tian, Yao, J., \& Wen, L. (2017).
    Collapse and earthquake swarm after North Korea's 3 September 2017 nuclear test.
    Abstract S43H-2968 presented at 2017 AGU Fall Meeting, New Orleans, LA, USA.
\item
    \Tian, \& Wen, L. (2017).
    Three types of Earth's inner core boundary.
    Abstract DI33B-0404 presented at 2017 AGU Fall Meeting, New Orleans, LA, USA.
\item
    Yao, J., \Tian, \& Wen, L. (2017).
    High-precision location, yield and tectonic release of North Korea's 3 September 2017 nuclear test.
    Abstract S43H-2967 presented at 2017 AGU Fall Meeting, New Orleans, LA, USA.
\item
    Yao, J., \Tian, Sun, L., \& Wen, L. (2017).
    Temporal change of seismic Earth's inner core phases: Inner core differential rotation or temporal change of inner core surface?
    Abstract DI33B-0405 presented at 2017 AGU Fall Meeting, New Orleans, LA, USA.
\item
    \Tian, \& Wen, L. (2017).
    Seismological evidence for a localized mushy zone at the Earth's inner core boundary.
    Presented at Gordon Research Conference: Interior of the Earth, South Hadley, MA, USA.
\item
    Yao, J., \Tian, Sun, L., \& Wen, L. (2017).
    Temporal change of seismic Earth's inner core phases: Inner core differential rotation or temporal change of inner core surface?
    Presented at Gordon Research Conference: Interior of the Earth, South Hadley, MA, USA.
\item
    \Tian, \& Wen, L. (2016).
    Seismic structures of the Earth's inner core boundary beneath the Bearing sea and Mexico.
    Abstract DI43A-2657 presented at 2016 AGU Fall Meeting, San Francisco, CA, USA.
\item
    \Tian, \& Wen, L. (2015).
    Varying seismic property of the Earth's inner core boundary.
    Abstract DI33A-2606 presented at 2015 AGU Fall Meeting, San Francisco, CA, USA.
\item
    \Tian, \& Wen, L. (2014).
    Seismic study on the properties of the Earth's inner core boundary.
    Abstract DI31B-4269 presented at 2014 AGU Fall Meeting, San Francisco, CA, USA.
\item
    Chen, X., \Tian, \& Wen, L. (2013).
    Seismic tracking of hurricane sandy.
    Abstract S11A-2296 presented at 2013 AGU Fall Meeting, San Francisco, CA, USA.
\item
    \Tian, \& Wen, L. (2013).
    Regional topography variation of Earth's inner core boundary.
    Abstract DI23A-2282 presented at 2013 AGU Fall Meeting, San Francisco, CA, USA.
\item
    Zhang, M., \Tian, \& Wen, L. (2013).
    A new method for earthquake determination: stacking multiple-station autocorrelograms.
    Abstract S51A-2301 presented at 2013 AGU Fall Meeting, San Francisco, CA, USA.
\item
    \Tian, \& Wen, L. (2012).
    Simulating wave propagation in a faulted medium using a 3D finite difference method.
    Abstract S43A-2458 presented at 2012 AGU Fall Meeting, San Francisco, CA, USA.
\end{etaremune}

\section*{Talks}
\begin{etaremune}
\item
	Global variations of Earth's 520- and 560-km discontinuities,
	Nanjing University,
	Jan. 7, 2021.
\item
	Global variations of Earth's 520- and 560-km discontinuities,
	Department of Earth and Space Sciences, Southern University of Science and Techbology,
	Nov. 27, 2020.
	\invited.
\item
    Global variations of the 520-km discontinuity.
    \textit{ 2nd Annual Earth and Environmental Sciences Student Research Symposium},
    Department of Earth and Environmental Sciences, Michigan State University, East Lansing, MI, USA.
    Feb. 23, 2019.
    \textbf{[5 minutes lightning talk]}
\item
    Collapse and earthquake swarm after North Korea's 2017 nuclear test.
    \textit{Institute of Geology and Geophysics, Chinese Academy of Sciences}, Beijing, China.
    Jun. 15, 2018.
\item
    Seismological evidence for a localized mushy zone at the Earth's inner core boundary.
    \textit{Institute of Geology and Geophysics, Chinese Academy of Sciences}, Beijing, China.
    Jun. 15, 2018.
    \invited
\item
    Fine-scale structure of the Earth's inner core boundary and aftershocks of North Korea's 2017 nuclear test.
    \textit{Institute of Earthquake Forcasting, China Earthquake Administration}, Beijing, China.
    Jun. 14, 2018.
\item
    Seismological evidence for a localized mushy zone at the Earth's inner core boundary.
    \textit{2017 Annual Meeting of Chinese Geoscience Union (CGU)}, Beijing, China.
    Oct. 17, 2017.
    \invited
\item
    Getting started with GMT in 60 minutes.
    \textit{Workshop on Analysis and Applications of Crustal Deformation Data}, Wuhan, China.
    Sep. 21, 2016.
    \invited
\item
    Seismic study on the properties of the Earth's inner core boundary.
    \textit{China Earthquake Networks Center}, Beijing, China.
    Jun. 30, 2016.
    \invited
\end{etaremune}

\section{Field Experience}

\begin{itemize}
\item \textbf{LEEP} (\textbf{L}ake \textbf{E}rie \textbf{E}arthquake ex\textbf{P}eriment),
      2018/10/12--2018/10/16, install 8 broadband seismic stations around Lake Erie.
\end{itemize}

\section*{开源软件}

\textit{*年份表示项目开始时间。所有软件目前仍在维护中。}

\begin{tabular}{p{0.05\textwidth} p{0.95\textwidth}}
2014 & \textbf{HinetPy} -- 用于从Hi-net网站申请和处理地震波形数据的Python软件包 \newline
       \url{https://github.com/seisman/HinetPy/} \\
\end{tabular}


\end{document}
